\chapter{Conclusioni}
\label{chap:conc}
Questo capitolo, conclusivo della tesi, si prefigge di illustrare i principali progetti futuri di Equiticket e le future direzioni dello sviluppo della soluzione software.
\section{Progetti e sviluppi futuri} \label{sec:prog}
%TODO
Tra i principali progetti di Equiticket sono annoverabili i seguenti: 
\begin{itemize}
\item Sviluppo di un \emph{misuratore fiscale} proprietario, in grado di gestire l'emissione dei titoli di accesso e la vendita congiunta di biglietti e merchandising (detta "\emph{cross-selling}"): l'obiettivo è quello di poter vendere sia i cosiddetti "bundle" (ovvero dei pacchetti che comprendono biglietto e merchandising/servizi, non modificabili), sia vendite dinamiche (biglietto combinato con merchandising/servizi a scelta, con gestione dinamica degli sconti). 
\item Sviluppo di una blockchain per emissione e rivendita di titoli di accesso, sul modello di BitTicket, così che possa essere offerto anche un servizio che protegga il consumatore dal possibile uso di bot per l'acquisto di grandi quantità di biglietti \cite{tackmann2017secure}. Una possibile architettura di partenza è stata descritta nella sezione \ref{block}.
\end{itemize}